\startproject project_hivataloslevel

\startluacode

  userdata = userdata or {}
  userdata.felado = {}
  userdata.cimzett = {}
  userdata.targymegjeloles = {}
  userdata.megszolitas = {}
  userdata.foszoveg = {}
  userdata.keltezes = nil
  userdata.udvozlesalairas = nil
  
  function userdata.beolvas()
    local file = io.open("level.txt", "r")
    local kezdo_allapotok = {"felado", "cimzett", "targymegjeloles", "megszolitas", "foszoveg"}
    local kezdo_allapot = 1
    local zaro_allapotok = {"keltezes", "udvozlesalairas"}
    local zaro_allapot
    local k, t
    local elozo_sor = ""
    for 
      sor in file:lines() 
    do
      k = kezdo_allapotok[kezdo_allapot]
      t = userdata[k]
      if 
        sor == ""
      then
        if 
          sor ~= elozo_sor
        then
          if
            kezdo_allapot < #kezdo_allapotok
          then
            kezdo_allapot = kezdo_allapot + 1
          end
          if 
            kezdo_allapotok[kezdo_allapot] == "foszoveg"
          then
            userdata.foszoveg[#userdata.foszoveg+1] = {}
          end
        end
      else
        if 
          k == "foszoveg"
        then
          t = userdata.foszoveg[#userdata.foszoveg]
        elseif 
          k == "targymegjeloles" and sor:sub(1,7) == "Tárgy:"
        then
          sor = "\\underlinee{Tárgy:} " .. sor:sub(8,#sor)
        end
        t[#t+1] = sor
      end
      elozo_sor = sor
    end
    for 
      zaro_allapot = 1, #zaro_allapotok
    do
      k = zaro_allapotok[#zaro_allapotok + 1 - zaro_allapot]
      userdata[k] = userdata.foszoveg[#userdata.foszoveg]
      table.remove(userdata.foszoveg, #userdata.foszoveg)
    end
  end

  function userdata.kiir_nemfoszovegszerut(tablanev, elso_sor, utolso_sor)
    local t = userdata[tablanev]
    elso_sor = elso_sor or 1
    utolso_sor = utolso_sor or #t
    t = {table.unpack(t, elso_sor, utolso_sor)}
    for 
      i, sor in ipairs(t)
    do
      context(sor)
      if 
        i < #t
      then
        context("\\break")
      end
    end
  end

  function userdata.kiir_foszovegszerut(tablanev)
    local t = userdata[tablanev]
    for 
      i, paragraph in ipairs(t)
    do
      for 
        j, sor in ipairs(paragraph)
      do
        context(sor)
        if 
          j < #paragraph
        then
          context("\n")
        end
      end
      if 
        i < #t
      then
        context("\n\n")
      end
    end
  end

\stopluacode

%\usemodule[languages-system]
%\loadinstalledlanguages
%\showinstalledlanguages

% Beállítom a körülvégott méretet lapméretnek.
\setuppapersize[A4]

\setuplayout[%
  header=1.5cc,%
  footer=3cc,%
  grid=yes,% Soregyen (Gyurgyák 319. o.).
  height=\dimexpr56cc+\headerheight+\headerdistance+\footerheight+\footerdistance\relax,%
  location=middle,%
  width=36cc,%
  backspace=\dimexpr0.5\dimexpr\paperwidth-\makeupwidth\relax\relax,%
  topspace=\dimexpr\dimexpr\paperheight-\makeupheight\relax/2\relax,%
  leftmargindistance=0.5cc,%
  rightmargindistance=0.5cc,%
  leftmargin=1.5cc,%
  rightmargin=1.5cc,
]

% Soregyennél a nagy ékezetes soroknál üres sorok keletkeznek,
% ha az alábbi nincs beállítva.
% (https://tex.stackexchange.com/questions/343299/when-i-activate-grid-layout-in-context-empty-lines-inexplicably-appear)
\setupinterlinespace[height=0.75,depth=0.25]

%% Oldalszámozás
\setupfootertexts[\doifelse\pagenumber1{\doifnot\lastpage1{\currentpage\,/\,\lastpage}}{\currentpage\,/\,\lastpage}]
\setuppagenumbering[location=]

% Betűkészlet
\setupbodyfont[libertinus,13pt]
%\setupbodyfont[13pt]
%\setupbodyfont[urwgaramond,13pt]

\mainlanguage[hu]
\language[hu]


\setuphyphenation[method=default]  % fontos, hogy így maradjun, különben elrontja a kivételeket
\setbreakpoints[compound]
\environment level_hyp

% Bekezdések
\setupindenting[no]
\setupwhitespace[line]
\setuptolerance[vertical,verytolerant]  % https://tex.stackexchange.com/q/264556/50554

% https://tex.stackexchange.com/a/320740/50554
\define[1]\underlinee{%
  \dontleavehmode\vbox to0pt{\vss
    \hrule height.4pt
    \vskip-\baselineskip \kern2.5pt
    \hbox{\strut\rlap{\color[white]{\pdfliteral{2 Tr 1.5 w}#1\pdfliteral{0 Tr 0 w}}}#1}
}}

\defineparagraphs[Zaroresz][
  after={\relax},
  before={\relax},
  n=2,
  location=depth,
]
\setupparagraphs[Zaroresz][1][width=0.5\textwidth]
\setupparagraphs[Zaroresz][2][width=0.5\textwidth, align=middle]




% Mutatja a szedéstükröt, ha nincs kikommentelve.
%\showframe
% Mutatja a rugalmas részeket, ha nincs kikommentelve.
%\showstruts
% Mutatja a soregyent, ha nincs kikommentelve.
%\showgrid[all]

\starttext

\ctxlua{userdata.beolvas()}

\startbodymatter

\startalignment[flushleft]

\noindentation
\ctxlua{userdata.kiir_nemfoszovegszerut("felado")}
%\emptylines[1]

\noindentation
\ctxlua{userdata.kiir_nemfoszovegszerut("cimzett")}
%\emptylines[1]

\noindentation
\ctxlua{userdata.kiir_nemfoszovegszerut("targymegjeloles")}
%\emptylines[1]

\stopalignment

\noindentation 
\ctxlua{userdata.kiir_nemfoszovegszerut("megszolitas")}
\blank

\ctxlua{userdata.kiir_foszovegszerut("foszoveg")}
\blank[\currentvspacing,samepage]  % https://tex.stackexchange.com/a/326445/50554
\startZaroresz
\ctxlua{userdata.kiir_nemfoszovegszerut("keltezes")}
\Zaroresz
\ctxlua{userdata.kiir_nemfoszovegszerut("udvozlesalairas", 1, 1)}\emptylines[2]
\inframed[frame=off,align=middle]{\cleaders\hbox to 0.25em{\hss.\hss}\hskip 5cm\zwj}\crlf
\ctxlua{userdata.kiir_nemfoszovegszerut("udvozlesalairas", 2)}
\stopZaroresz

\stopbodymatter

\stoptext

\stopproject
